\chapter{Related Work}\label{chap:RelatedWork}

Our work on an index token for passive asset management is at the intersection of Finance, Computer Science and Cryptography, and therefore, the thesis is based on many the previous works in those field studies with the emphasis on the prominent ones on Blockchain technology and Ethereum with examples including \cite{satoshi2008peer}, \cite{wood2014ethereum}, \cite{buterin2014next}. However, we consider the works that this thesis and our system prototype is established upon as the foundations and present them in chapter \ref{chapter:Background}. On the other hand, there are DeFi projects that also implement the concept of an index token for passive asset management and are the direct alternatives to the system we plan to develop. In the remaining of this chapter, we will select and discuss some of the most noticeable DeFi Index tokens on the current market based on their market capitalization from Coingecko \cite{geckoDefiIndexByCap} and the availability of their publication papers including white papers and yellow papers.


\section{The Set Protocol}

At present, the most prominent instances of DeFi Index tokens on the market are DeFi Pulse Index (DPI) and ETH 2x Flexible Leverage Index (ETH2X-FLI) with their market capitalization being \$201,355 and \$111,974 with therefore, being the two largest Index tokens by market capitalization not only on Ethereum, but also across different blockchains \cite{geckoDefiIndexByCap}. The commonality that both DPI and ETH2X-FLI share is the Set Protocol underpinning their systems \cite{tokensetsExplore}. The goal of the Set Protocol \cite{setprotocolwhitepaper} is to simplify the asset management process on the crypto-space. To achieve this goal, the protocol provides a factory smart contracts that allows anyone to create their own index tokens following the Set standard and thereby become on-chain asset managers. From the perspective of a new manager who wishes to open their own index fund on Ethereum, the creation of a Set contract is a 3-step process: selecting the underlying tokens, setting metadata for the Set contract, publishing the new Index fund. After that, new manger can manage the Set by allowing investors to invest or redeem their capital. 


\section{Crypto20}

The Crypto20 tokens is a  \textit{closed-hybrid fund} \cite{schwartzkopff2017crypto20} with an ICO in 2017 and no further issuance of tokens is possible after the ICO. Therefore, the trades of the C20 token on exchanges do not affect the token price. The portfolio of Crypto20 comprises 20 largest crytocurrencies and tokens in term of capitalization across different chains, however, the C20 token is on the Ethereum blockchain. The portfolio is maintained with price floor of 10\% by a trusted party for rebalancing strategies and portfolio update.