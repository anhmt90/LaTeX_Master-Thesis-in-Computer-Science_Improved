\chapter{\abstractname}

%TODO: Abstract

With the growth of cryptocurrencies and crypto-assets in recent years, investors now have a new investment area to pursue profits. A considerable part of this picture is attributed to the Ethereum blockchain with the ability to run Turing-complete scripts called smart contracts and allow everyone to create their own crypto-tokens serving myriad purposes, thereby, also opening new investment opportunities to different interested parties. While investing in individual tokens is easy at first, investors will be, however, less exposed to the market and actively managing various invested tokens at the same time is not possible for everyone, especially for new investors, who do not have the expertise in the Decentralized Finance (DeFi) market as well as the technical knowledge required to understand the smart contracts and the protocols underlying the tokens. For such reasons, we suggest the development of an index token acting as a portfolio comprising many different underlying component tokens. By investing in an index token, investors effectively invest and own the proportional amounts of all the individual components constituting the index token, thereby, achieving diversified market exposure.

In this thesis, we discuss our prototype for the realization of such an index token using smart contracts written in the Solidity language. We name this index token DFAM, which stands for \textit{DeFi Asset Management}, and require that all the underlying component tokens belong to the same token category of "Decentralized Lending, Saving and Asset Management". To identify the category of tokens, we resort to the ITSA's International Token Classification (ITC) framework.

Furthermore, we also design and implement, as a proof-of-concept, a \textit{decentralized index fund} that can function by itself on the Ethereum blockchain to serve investors for activities around their investments into the index fund. To enable this automation, the index fund relies on a decentralized exchange (DEX) for the source of acquiring the underlying tokens when receiving ETH from investors' investment capital and Uniswap is our DEX of choice for this prototype.

Our prototype has proved to be working on a real-life blockchain on the Ropsten testnet with the most basic functionalities that a conventional index fund can offer such as handling investment, redemption, portfolio update and portfolio rebalancing. However, there is still much room for further development on the prototype and further considerations on the economic perspective are also needed before the entire system can be brought into production as a real DeFi application on the Ethereum Mainnet. 








\makeatletter
\ifthenelse{\pdf@strcmp{\languagename}{english}=0}
{\renewcommand{\abstractname}{Kurzfassung}}
{\renewcommand{\abstractname}{Abstract}}
\makeatother

% \chapter{\abstractname}

%TODO: Abstract in other language
% \begin{otherlanguage}{ngerman} % TODO: select other language, either ngerman or english !

% \end{otherlanguage}


% Undo the name switch
\makeatletter
\ifthenelse{\pdf@strcmp{\languagename}{english}=0}
{\renewcommand{\abstractname}{Abstract}}
% {\renewcommand{\abstractname}{Kurzfassung}}
\makeatother